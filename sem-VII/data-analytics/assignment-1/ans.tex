\documentclass[a4]{article}
\usepackage{listings}


%opening
\title{Assignment 01}
\author{Vivek Rai, 12BT30025}

\begin{document}
\lstset{language=SQL}
\maketitle

\section{SQL Queries}
\subsection{Ans}
\begin{lstlisting}
SELECT product.pid, product.name
FROM product, inventory
WHERE product.pid = inventory.pid AND inventory.stock < 5;
\end{lstlisting}

\subsection{Ans}
\begin{lstlisting}
SELECT DISTINCT supplier.sid, supplier.name
FROM supplies
JOIN manufactures USING(pid)
JOIN supplier USING(sid)
JOIN manufacturer USING(mid)
WHERE manufacture.name = 'manufacturer_2';
\end{lstlisting}

\subsection{Ans}
\begin{lstlisting}
SELECT product.pid, name, inventory.stock
FROM product, inventory
WHERE product.pid = inventory.pid
ORDER BY stock DESC;
\end{lstlisting}

\subsection{Ans}
\begin{lstlisting}
SELECT pid, name, COUNT(*) as `num`
FROM product
JOIN supplies USING(pid)
GROUP BY pid HAVING num = 1;
\end{lstlisting}

\subsection{Ans}
\begin{lstlisting}
SELECT pid, product.name
FROM product
JOIN inventory USING(pid) 
WHERE stock = (SELECT MIN(stock) FROM inventory);
\end{lstlisting}

\subsection{Ans}
\begin{lstlisting}
SELECT sid, name, COUNT(*) as `product_supplied`
FROM supplier
JOIN supplies USING (sid)
GROUP BY sid;
\end{lstlisting}

\subsection{Ans}
\begin{lstlisting}
SELECT manufactures.mid, manufacturer.name, AVG(min_age)
FROM manufactures
JOIN product USING(pid)
JOIN manufacturer USING(mid)
GROUP BY mid 
ORDER BY min_age DESC
LIMIT 1;
\end{lstlisting}

\subsection{Ans}
\textbf{Which manufacturer has highest throughout?}
\begin{lstlisting}
SELECT mid, name, COUNT(*) AS `products_produced`
FROM manufacturer
JOIN manufactures USING(mid)
GROUP BY mid
ORDER BY `products_produced` DESC
LIMIT 1;
\end{lstlisting}

\section{Integrity Constraint}

\subsection{Ans}

After specifying primary keys for the tables, adding further values requires 
the value corresponding to primary key to be unique, else the operation is 
rejected.

For example, if we try to insert into \textup{product} table, a new tuple with
pre-existing \textup{pid} but new \textup{name} or \textup{min\_age}, it will 
be 
rejected. Without this constraint,however, it will be added perfectly to the 
table.

\subsection{Ans}
MySQL database schema when used with foreign keys, allows \textit{referential 
integrity}. It provides a flexible way of reducing data duplication, prevent 
changes from happening or propagating changes across different tables with 
minimum action.

In case we specify a foreign key relation between two tables, and try to 
perform a \textbf{INSERT} or \textbf{UPDATE} operation on the child table 
corresponding to which there is no entry in parent table, the operation is 
rejected. For example, the situation of inserting a product in inventory table
corresponding to which no entry exists in Product table.

This obviously depends on the parent-child relation and the way foreign-key 
constraints are specified. In case a match is found, further behavior is 
controlled by a different set of constraints, \textbf{CASCADE}, 
\textbf{RESTRICT}, for example.

Not having these constraints will allow different inconvenient situations where 
we have data duplication, manual effort of adding/updating data to each table 
and ensuring that they are correct etc.,

\end{document}
