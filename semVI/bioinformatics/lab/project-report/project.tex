\documentclass[12pt,a4paper]{report}
\usepackage{hyperref}
\usepackage{gensymb}
\usepackage{listings} 
\usepackage[utf8]{inputenc}
%...
\usepackage[letterpaper, margin=.9in]{geometry}
% Title Page
\title{Bioinformatics Lab Manual \\ \& \\ Class Notes}
\author{Vivek Rai \\ \textnormal{12BT30025}}

\begin{document}
\maketitle

\chapter*{Primer Design}

Primer design has important applications in PCR, DNA sequencing, and 
hybridization; thus at the core of almost all biological activity today.\\

\textbf{Objective:} To understand the process of PCR, importance of primers, 
considerations involved in designing a primer culminating with a hands on 
session on actual design of regular and degenerate primers.\\

\noindent
\textbf{Primers} are short length (12-20 bp) oligonucleotides which serve as 
the starting point for DNA synthesis. It is required for DNA replication 
because the enzymes that catalyze this process, DNA polymerases, can only add 
new nucleotides to an existing strand of DNA.This, however, reveals a very 
important limitation of primers that we need to 
have at least some information about the target sequence. In case, the sequence 
is exactly known we can easily design a primer based on general considerations 
but otherwise a \textit{degenerate} primer has to be designed with more general 
target (which in turn reduces the reliability of amplification).\\

\textbf{General considerations:}
\begin{itemize}
    \item Melting temperature ($T_m$) between 55 and 65\degree C (usually 
    corresponds 
    to 45-55\% G+C for a 20-mer)
    \item Absence of dimerization capability.
    \item Absence of significant hairpin formation (usually $>$ 3 bp).
    \item Lack of secondary priming sites in the template.
    \item Low specific binding at the 3' end, to avoid mispriming.
    \end{itemize}

\noindent
\textbf{Degenerate Primers} are used to amplify DNA when the target sequence is 
not accurately known. In this case, we compromise with the target specificity 
by synthesizing several primers, each having different nucleotide at a 
particular position. For example, if the protein sequence is known and we want 
to amplify the corresponding genes, then our primer should include the bases 
matching to \textit{conserved regions} in protein, and If some of the residues 
are not completely conserved, then the oligo sequence will need to accommodate 
all possible codons of all amino acid residues at that site

\clearpage
\section*{Laboratory Session}

\subsection*{Regular Primer Design}
\begin{enumerate}
    \item Obtain FASTA sequence for a nucleotide (AF103742) from NCBI 
    nucleotide database.
    \item We then use \textup{Primer3Plus} and OligoPerfect tools available 
    online to design our Primer by specifying the downloaded sequence, and 
    parameters taken into concern from previous page (i.e., melting temperature,
    GC content etc.,)
    
    \textbf{Thermodynamics process -} SantaLucia 1998\\
    \textbf{Salt correction formula -} SantaLucia 1998\\
    \textbf{Max-PolyX -} 3\\
    \textbf{Max $T_m$ difference -} 5\degree C
    
    The optimal values of an annealing temperature, GC content, $T_m$ can be 
    calculated using established formula.
    \item We also practiced a the design of a primer for cloning experiment by 
    incorporating suitable restriction sites by referring to vector map of a 
    plasmid and cross checking target sequence to be cloned for restriction 
    sites using \textup{NEBCutter}.
\end{enumerate}

\subsection*{Degenerate Primer Design}
\begin{enumerate}
    \item We try to design a degenerate primer for a protein called 
    \textit{sericine},
    usually found in silk-producing organisms.
    \item From the NCBI website we retrieve a couple of protein sequences
    for the protein of interest and download them in `sequence.fasta` file.
    
    We ensure that the protein sequences are within 600-900 aa in length.
    \item A clustalW alignment is performed and obtained multiple sequence 
    alignment is saved in file `clustalw-alignment.clustalw`.


Now we examine the start and end sequences of obtained alignment file.

The alignment start for similar nature of organisms is: MRFVLCC\\
The alignment end for similar nature of organisms is  : LRKNIGV

As we can see, both of these short sequences contain L, R; the amino acids
that we tend to avoid in primer design (because of degeneracy). For this, we
consult different online databases that enlist the codon usage pattern in
the organism of our interest and try to locate the pattern.

This information can help us design more specific primers (although primer
is intended to be degenerate, we don't want the degeneracy to be too high to
reduce the specificity).

\item Corresponding to both these peptide sequences, we design an 
oligonucleotide,
accounting suitably for degenracy.

For example,

For forward primer, MRFVLCC

M - ATG\\
R - (A + C)G(A + T + G + C)  = MGN\\
F - TT(C + T)                = TTY\\
V - GT(A + T + G + C)        = GTN\\
L - (C + T)T(A + T + G + C)  = YTN\\
C - TG(C + T)                = TGY\\

Complete sequence: \textbf{ATGMGNTTYGTNYTNTGY}

For reverse primer, LRKNIGV

R - (A + C)G(A + T + G + C)  = MGN\\
K - AA(A + G)                = AAR\\
N - AA(C + T)                = AAY\\
I - AT(A + T + C)            = ATH\\
G - GG(A + T + G + C)        = GGN\\
V - GT(A + T + G + C)        = GTN\\

Complete sequence: \textbf{MGNAARAAYATHGGNGTN}

Reverse complement: \textbf{NACNCCGATRTTYTTNCK} (Final primer)

Here we can note that R unnecessarily introduces a degeneracy in form of N. 
Hence,
we can drop that amino acid without any harm.
\end{enumerate}

\textbf{NOTE:}
\begin{itemize}
\item While designing primers, ensure that you avoid L, R, S amino acids as 
much as
possible. Simply because they are highly degenerate and are coded by 6 codons.

\item We try to avoid degeneracy at the 3' end of the sequence because that is 
where
polymerase starts the synthesis. Non degeneracy would ensure correctness and
accurate PCR product.
\item We also take into account any restriction site, if used, by adding 2-3 
bases upstream of primer since many restriction enzymes require a few  bases to 
\textit{sit}.

\end{itemize}
%\part*{Primer Design}
\chapter*{Sequencing}
%\phantomsection
%\addcontentsline{toc}{chapter}{Primer Design}
\textbf{Question 1.} Why it is called the shotgun genome sequencing?
\subparagraph{}
In this sequencing method, a DNA is broken randomly into a large number of 
small fragments (50-200 Kb) and then sequenced using \textit{chain termination} 
methods to obtain reads. Since, this appears analogous to shooting down an 
object and breaking into the pieces, hence the name.\mbox{}\\

\noindent
\textbf{Question 2.} Why is it called NGS?
\subparagraph{}
Sanger sequencing was one of the first sequencing methods that 
was used to obtained a \textbf{finished genome} of an organism. The process, 
however, was very expensive and time consuming. Later, newer technologies were
invented which made a major depart from original approach and allowed large 
scale sequencing at inexpensive rates. The massively parallel nature of these 
methods along with highly automated method allowed for rapid progress and whole 
genome sequencing of many organism in succession.

Because of these major changes, the newer technologies were called \textit{next 
generation sequencing} platforms.\mbox{}\\

\noindent
\textbf{Question 3.} What is the function of the variability of insert size?
\subparagraph{} Insert size refers to the length of DNA sequence between the 
two primer ends of a template. If the insert size is large and constant, that 
fragment will never be completely sequenced (or require a large number of 
attempts) because there will be a gap left in the between. This introduces 
unreliability in the results which can be avoided by having variable insert 
sizes. It will ensure that a large number of 
fragments are sequenced completely and posses overlapping reads, which in turn 
allow us to be stitch the sequences with more confidence, thereby improving 
sequence coverage.\mbox{}\\

\noindent
\textbf{Question 4.} What is called bridge amplification and why is it required?
\subparagraph{} \textit{Bridge amplification} is a solid-phase template 
amplification technology used by Illumina/Solexa sequencing platform. In this 
method, universal primers (both forward and reverse) are immobilized on a 
surface and DNA templates (ligated with similar primers) are allowed to 
hybridize in the presence of polymerase. These complimentary primers of DNA and 
the surface anneal partly to immobilize the template as well. The DNA template 
then forms a bridge with the immediately adjacent primer (using the other free 
end) and is amplified to form a cluster of the same template.\mbox{}\\

\noindent
\textbf{Question 5.} How Illumina technique is unique compared to other 
technologies?

\subparagraph{} Illumina/Solexa sequencing platforms just use DNA polymerase
thereby avoiding the need of multiple reagents, use fragmented solid-phase 
template amplification, and is one of the most widely used and well supported
platforms.\mbox{}\\

\noindent
\textbf{Question 6.} What is the different between FASTA and FASTQ files? What 
comes with what function?
\subparagraph{} FASTA is a file format for storage and transfer of nucleotide 
and protein bases. FASTQ on the other hand, although similar to FASTA files, is 
generated by sequencing platforms and contains the raw reads along with their 
corresponding quality scores.\mbox{}\\\\
\textbf{FASTA format:}
\begin{verbatim}
>gi|39748133|emb|BX571963.1| Rhodopseudomonas palustris CGA009 complete genome
ATCGGTCGAGGCGAAATCTTCACCCTGCCCTCGGAATCATATCCATTGCAGCGGAGGGGCCGTCGTGGTT
TTCATAGTCCACCCGCGACGCCCACGGCTCTTCAGATCAGCGCGGTTTGAGAACCAAGGGCGGACATGCA
\end{verbatim}
\textbf{FASTQ format:}
\begin{verbatim}
@HWUSI-EAS300R_0005_FC62TL2AAXX:8:30:18447:12115#0/1
CGTAGCTGTGTGTACAAGGCCCGGGAACGTATTCACCGTG
+HWUSI-EAS300R_0005_FC62TL2AAXX:8:30:18447:12115#0/1
acdd^aa_Z^d^ddc`^_Q_aaa`_ddc\dfdffff\fff
\end{verbatim}
\noindent
\textbf{Question 7.} What is generated after the assembly of the sequence reads?
\subparagraph{} A contig.\mbox{}\\

\noindent
\textbf{Question 8.} How specific function of a stretch of a nucleotide can be
predicted?
\subparagraph{} Predicting the role and function of a stretch of nucleotide is 
one of the popular problems in bioinformatics called \textit{gene annotation}. 
It may often require assistance and verification from experimental results. \\ 
In general, if nothing is known about the sequence, one can start by BLASTing 
the sequence and finding if there are already annotated sequences available. 
Based on the results obtained, on can further investigate the presence of
conserved regions, promoter sites etc.,\mbox{}\\

\noindent
\textbf{Question 9.} Why adapters are required but primers are not?
\subparagraph{} Primers are, in general, used when the sequence of the target 
DNA is already known. In this case, however, we are un aware of the sequence of 
target DNA - in fact, that itself is the goal of the experiment. Thus, 
sequencing platforms use \textit{adapters} during the \textit{template 
amplification} process which is ligated to both of ends of DNA and can anneal 
to the universal primers. This ensures that complete DNA is amplified and no 
bases are lost due to primer removal.\mbox{}\\

\noindent
\textbf{Question 10.} How can a person do multiple samples at a time? Can it be 
done?
\subparagraph{} Yes, multiplexed sequencing can be done and is readily 
supported by several sequencing platform providers. In this method, DNA 
libraries are \textit{tagged} with a unique identifier, or index, during the 
sample preparation. Multiple samples are then pooled into a single sample, 
analyzed, and segregated based on the identifiers or tags.\mbox{}\\

\chapter*{QIIME}
\textbf{Question 1.} What are the major advantages of QIIME over the other rRNA 
analysis tools?
\subparagraph{}
QIIME (pronounced \textit{chime}) is one of the recent 
developed, open-source bioinformatics pipeline for performing microbiome 
analysis from raw DNA sequencing data (Caporaso \textit{et al}, 2010). Compared 
to other available rRNA analysis tools, it offers distinct advantages such as:
\begin{itemize}
    \item{Ability to read and process large amount of genomic data generated 
    from next generation sequencing methods,}
	\item{Ability to run and access multiple software at a time,}
    \item{Support for various file formats, demultiplexing and quality 
    filtering, OTU picking, taxonomic assignment, and phylogenetic 
    reconstruction, and diversity analyses and visualizations,}
	\item{Cross platform, easy to use, open-source and free of cost}
\end{itemize} \mbox{}\\


\noindent
\textbf{Question 2.} What are the pre-processing steps in QIIME?
\subparagraph{} QIIME has several preprocessing data filters like -
\begin{itemize}
    \item Primer removal
    \item Demultiplexing
    \item Denoising, Quality filter and Chimera Checking
\end{itemize}\mbox{}\\

\noindent
\textbf{Question 3.} What are the different methods of OTU picking?
\subparagraph{} QIIME provides three ways of OTU picking namely, \textit{de 
novo}, closed-reference, and open-reference OTU picking.
\begin{itemize}
    \item {\textit{De novo:} reads are clustered against one another without 
    any external reference sequence collection.}
    \item {closed-reference: reads are clustered against a reference sequence 
    collection and any reads which do not hit a sequence in the reference 
    sequence collection are excluded from downstream analyses.}
    \item {open-reference: reads are clustered against a reference sequence 
    collection and any reads which do not hit the reference sequence collection 
    are subsequently clustered de novo.}
\end{itemize}\mbox{}\\

\noindent
\textbf{Question 4.} Compare $\alpha$ and $\beta$ diversity?
\subparagraph{} $\alpha$ and $\beta$ diversity were coined by R. H. Whittaker 
to 
describe the total species diversity in a landscape. Any such description 
according to him, requires describing different species and their interaction 
at two levels - local or habitat and global.
\begin{itemize}
    \item {$\alpha$-diversity means the diversity of the community within one 
    site (or one sample), i.e., the number of species and their proportion 
    within one sampling site.}
    \item {$\beta$-diversity means the dissimilarity between communities of two
    sites (or two samples) due gain or loss of species from replacement or 
    biotic changes along environmental gradients.}
\end{itemize}\mbox{}\\

\noindent
\textbf{Question 5.} What are the different parameters required for the quality
filtering?
\subparagraph{} Quality filering in QIIME is done using USEARCH, which is a 
sequence analysis tool. The quality of raw reads after filtration depends on 
many parameters such as -

\begin{itemize}
    \item {Expected Errors (E): Discarding reads based on the total expected 
    error for all bases in the read.}
    \item {Truncation Length (L): Truncating sequences at the $L^{th}$ base 
    and 
    discarding if the length is less than L.}
    \item {Quality score (N): Discarding all reads having quality score $< = 
    N$.}
	\item {Strip Length (L): Number of sequences to strip (delete) from 
	starting of the sequence}
\end{itemize}\mbox{}\\

\noindent
\textbf{Question 6.} What are the tools used for aligning the sequence, 
assigning taxonomy in de novo OTU picking?

\subparagraph{} QIIME uses MUSCLE for sequence alignment (pyNAST is a suitable 
alternative) and Uclust Consensus Taxonomy classifier for assigning taxonomy in 
\textit{de novo} OTU picking.\mbox{}\\

\noindent
\textbf{Question 7.} Write down the workflow of QIIME for analyzing the ecology 
of environmental samples through 16S rRNA amplicon?

\subparagraph{} The workflow consists of the following steps:
\begin{itemize}
    \item \textbf{Obtaining the data}
    \item \textbf{Validate the mapping file}
    \item \textbf{Demultiplex and quality filer reads}
    \item \textbf{OTU picking}
    \item \textbf{Summarize communities}
    \item \textbf{Diversity indexes}
\end{itemize}\mbox{}\\

\noindent
\textbf{Question 8.} What is the necessity of microbial diversity study?
\subparagraph{} Microbes play an inevitable role in almost every sphere of life 
- ranging from the deep trenches of the oceans to freezing glaciers, from hot 
and extreme conditions of hot springs to the gut of animals. They carry about
important processes in the nature and are present at every imaginable place on
the Earth. These constant biotic and abiotic challenges have equipped microbes
with a very rich set of biochemical activity (by virtue of their genes). Their 
characterization, study, and analysis of diversity allows us to understand 
complex and novel processes, and in some ways harness that knowledge for our 
intellectual and economical use.\mbox{}\\

\noindent
\textbf{Question 9.} Why is QIIME needed?
\subparagraph{} QIIME encompasses diversity analyses tools and techniques from 
several areas which otherwise are fragmented and difficult to collect. Using 
QIIME, one can process his entire collection of raw reads and obtain final 
results with minimal number of steps.

\chapter*{Phylogeny}
The study of evolutionary relationship.

All the organisms on earth have descended from a common
ancestor, which means that set of species, extant or extinct,
is related. This relationship is Phylogeny, and the study of existing
organisms' morphological, physiological, and molecular characteristics
to create a tree -- is called Phylogenetics. \textbf{Molecular Phylogenetics 
}uses the 
structure and function of the molecule and how they change over the time to 
infer relationships.

\textbf{Phylogenetic analysis }is the process of testing hypotheses about the
descent of species from a common ancestor.The null hypothesis in most cases is 
that the organisms have a common
ancestor (it is because it is natural to accept the fact that any two
organism must have been related)

\section*{Laboratory session}

\textbf{Objective:} 16s rRNA Analysis to determine the relationship of an 
unknown sequence.\mbox{}\\

\begin{enumerate}
	\item We get the unknown sequence and BLAST it against three databases 
	(NCBI,RDP, Ez Taxon).
    \item We take the first five results from all three blasts
and put it in single file. We also add a sequence from Archeabacteria
to act as a negative control.
	\item We then use \textbf{MEGA }to generate tree.
    \item Output files are then saved and submitted to faculty.
\end{enumerate}
\end{document}          
